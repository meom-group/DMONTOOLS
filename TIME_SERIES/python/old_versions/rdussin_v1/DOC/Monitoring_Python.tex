\documentclass[a4paper,11pt]{article}
% Pour l'utilisation des caractères accentués :
%\usepackage[latin1]{inputenc}
% Pour la modification de graphiques :
\usepackage{epsfig}
%\usepackage{includegraphics}
\usepackage{graphicx}
\textwidth = 16.5cm
\textheight = 23.5cm
\topmargin = -1cm
\oddsidemargin = -0.5cm
\evensidemargin = -0.5cm
% 
 
\begin{document}

\title{Time series with python. How-to ?}
%-------------------------------------------------------------------
%-------------------------------------------------------------------
\author{Raphael Dussin  
\thanks{ Laboratoire de Physique des oc\'{e}ans, CNRS-IFREMER-UBO, Plouzan\'e, France}}
\date{January 2009}
\maketitle

\section{Introduction}

This document presents the time-series part of the monitoring. This step is very important as it is the major tool to monitor
the runs in almost real-time (as the 2D monitoring is much longer at present time). We describe here new tools for monitoring.
Nowadays, those time-series are made using a set of matlab scripts (MONITOR\_MATLAB) and this implies firstly to have a matlab licence
(which is usually the case in research labs) and secondly to have someone running the time-series script on one of the lab computers
after each year of the run finishes. The idea raised to perform those time-series with free tools such as python (which are available or
can be installed in most of the computational centers which refuses to buy licences) in order to compute the time-series automatically
after the run's current year has finished.  

\section{Install}

First, check you have python 2.5 or newer and the matplotlib library. Then copy MONITOR\_PY.tar located in
\begin{verbatim} ~rcli600/SHARE \end{verbatim} in your \$HOME directory and decompress using tar -xvf. 
No compilations are needed as it is just a collection of scripts.


\section{How it works}

The \textbf{monitor\_python.skel.ksh} should be edited by your production machine script at the end of the computation of the mtl files
(after script \textbf{mkmtl.ksh} on storage machine gaya). 
For example :
\begin{verbatim}
rsh $ploting_machine "cd $HOME/MONITOR_PY ; \
		cat monitor_python.skel.ksh | sed -e "s/CCOONNFFIIGG/$CONFIG/g" -e "s/CCAASSEE/$CASE/g" \
		 > ./RUN_$CONFIG-$CASE/monitor_python.${CONFIG}-${CASE}.ksh "
\end{verbatim}

\noindent
Then execute \textbf{./monitor\_python.CONFCASE.ksh} on plotting machine.
\begin{verbatim}
rsh $plotting_machine "cd $HOME/MONITOR_PY/RUN_$CONFIG-$CASE ; \
                chmod 744 ./monitor_python.${CONFIG}-${CASE}.ksh ; \
                ./monitor_python.${CONFIG}-${CASE}.ksh"
\end{verbatim}

\noindent
This script edits the python files in the SKELS directory, puts the resulting files in a custom \textbf{RUN\_CONFCASE} directory and sequentially execute them.
The computations are quickly done, less than one minute on a personal linux laptop.
Data from observations are also provided in the \textbf{DATA\_obs} directory. You can the automatically add the figures on your web site using a rcp command at the end of \textbf{monitor\_python.skel.ksh}
 


\section{Developement strategy}

The python scripts are intended to read mtl files and have high sensibility to their content. However, some eforts were done to make
the scripts more flexible so that adding or deleting variables in the mtl file does not induce a failure (this is true for some scripts and need further work).
The reason why it is important is because we run several kind of configs (from global to regional) so part of the data given by a global run will not be written
in a mtl file in a North Atlantic config. The architecture of the scripts is :

\begin{enumerate}
\item drakkar environnement
\item mtl file reading
\item default settings 
\item automatic config selection w/ custom settings
\item data manipulation
\item plot cosmetics settings
\item plot general commands
\item figures saving
\item dev notes

\end{enumerate}

\section{To do :}

\begin{enumerate}
\item distribution with SVN.
\item cosmetic tuning
\item al the scripts are not 100 \% flexible
\end{enumerate}

\end{document}
